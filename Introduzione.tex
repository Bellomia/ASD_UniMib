\section{Introduzione e nota sui linguaggi di programmazione utilizzati}

La seguenta raccolta di esercizi svolti costituisce buona parte della prova d'esame per il corso di \emph{Analisi Statistica dei Dati} presso il Dipartimento di Fisica dell'Università degli Studi di Milano-Bicocca. I sette esercizi assegnati spaziano grosso modo tra tutti gli argomenti affrontati durante tale corso semestrale e rappresentano esempi applicativi delle tecniche illustrate a lezione. Gli ambiti in cui tali applicazioni si inseriscono sono piuttosto variegati, da esercizi molto formali e astratti, quali la ricerca di minimi globali di celebri funzioni "patologiche" o il calcolo di integrali per mezzo di tecniche Montecarlo, a vere e proprie analisi dati su esperimenti (simulati) di ottica e fisica delle particelle. Non manca naturalmente una buona palestra su tutti quei fondamenti di matematica numerica e scienza dei calcolatori che costituiscono il necessario precedente (tecno)logico per qualsiasi moderna pratica di analisi statistica dei dati.\\

\noindent La natura intrinsecamente multidisciplinare della materia trattata nonché la marcata disomogeneità che caratterizza i differenti percorsi di una laurea magistrale in fisica fanno sì che la modalità di svolgimento degli esercizi sia sostanzialmente molto libera nella scelta degli approcci e degli strumenti di cui avvalersi. Personalmente ho dunque scelto di utilizzare per lo svolgimento di tutti gli esercizi dei linguaggi di programmazione \emph{interpretati} e con una sintassi di \emph{alto livello} . Tale scelta è chiaramente mirata alla flessibilità e all'immediatezza di scrittura che a mio parere un'attività di lavoro così eterogenea intrinsecamente richiede.  In particolare ho implementato quasi tutto in \texttt{Phyton}, linguaggio piuttosto moderno ma ormai caro agli ambienti della ricerca scientifica, con il vezzo di qualche incursione nel mondo di \texttt{R}, habitat ideale per qualunque applicazione di statistica e \emph{data science}. Maggiori informazioni possono essere facilmente reperite nei rispettivi siti ufficiali: \url{www.python.org} e \url{www.r-project.org}, fermo restando che la documentazione specifica di eventuali funzioni e pacchetti particolari sarà citata nel corpo della presentazione, sotto forma di nota a piè di pagina (con link). Disseminati lungo il testo saranno anche i listati di tutti gli \emph{script} implementati, con specificato il linguaggio e evidenziata opportunamente la relativa sintassi. L'indice dei codici listati può essere consultato a pagina~\pageref{listoflistings}.\\

\bigskip

\noindent Trieste, \today\\

\newpage